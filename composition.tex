\documentclass[12pt]{article}
\usepackage[T2A]{fontenc}
\usepackage[utf8]{inputenc}
\usepackage[russian]{babel}
\usepackage{amsmath}
\usepackage{amssymb}
\setlength{\parskip}{1.5em}
\usepackage[left=2cm, right=2cm, top=2cm, bottom=2cm, bindingoffset=0cm]{geometry}
\newtheorem{problem}{Задача}
\newcommand\TheSolution{%
  \textbf{Решение.}\\%
}

\begin{document}
    \begin{problem}
        На празднике CSC Дима нашёл 10 одинаковых наклеек с логотипом центра. Для себе он взял только одну. Оставшиеся он решил распределить между своими друзьями: Иваном, Антоном, Олегом и Андреем. Сколько способов раздать наклейки?
    \end{problem}
    \TheSolution
    Диме требуется раздать оставшиеся 9 наклеек своим 4 друзьям.

    Рассмотрим простой случай: Дима отдал все наклейки одному своему другу. Этот вариант дает 4 способа раздачи наклеек.
    
    Пусть теперь Дима раздаст все наклейки ровно двум своим друзьям, причем каждый из них получит как минимум одну.
    Рассмотрим уравнение $x + y = n$, где $x, y \in \mathbb{N}$ — количества наклеек, которые Дима подарил первому другу и второму соответственно, а $n \in \mathbb{N}$ — собственно число наклеек.
    Нетрудно видеть, что $x, y \leqslant n - 1$, поскольку в противном случае такое уравнение, очевидно, решений не имеет.
    Заметим, что $y = n - x$, то есть между $y$ и $x$ установлено взаимно однозначное соответствие. Это приводит к тому, что при фиксированном $x$, удовлетворяющем условиям этого случая, $y$ будет определен однозначно, аналогично фиксированный $y$ однозначно задает $x$. Перебирая все значения $x \in \{1, 2, ..., n - 1\}$, получаем значения $y$, удовлетворяющие условиям этого случая: $y \in \{n - 1, n - 2, ..., 1\}$.
    То есть это уравнение имеет $n - 1$ решений, в нашем случае $9 - 1 = 8$ решений. Осталось заметить, что такое уравнение можно записать для любых пар друзей Димы, которых всего $C_4^2 = 6$. И каждая пара друзей Димы будет имеет по 8 вариантов распределения наклеек, значит этот случай дает нам $6\cdot8 = 48$ способов.

    Положим теперь, что Дима решил раздать наклейки ровно 3 своим друзьям, причем каждый из них получил как минимум одну наклейку.
    Рассмотрим уравнение $x + y + z = n$, где $x, y, z \in \mathbb{N}$ — количества наклеек, которые Дима вручил первому, второму и третьему другу соответственно, а $n \in \mathbb{N}$ — число доступных наклеек.
    По тем же причинам, что и в предыдущем случае, имеем $x, y, z \leqslant n - 2$. Рассматриваемое уравнение эквивалентно уравнению $x + y = n - z$.
    Заметим, что если зафиксировать $z$, то задача сводится к предыдущему случаю, и решений будет всего $n - z - 1$.
    Тогда в зависимости от значения $z \in \{1, 2, ..., n - 2\}$ имеем соответствующие количества решений: $n - 2, n - 3, ..., 1$. В сумме их $1 + 2 + ... + n - 2 = (n-2)(n-1)/2$. Для случая $n = 9$ имеем число решений, равное 28. Такое уравнение можно составить для всех $C_4^3 = 4$ троек друзей Димы. Значит этот случай дает нам суммарно $4\cdot28 = 112$ способов распределения наклеек.

    В последнем нетривиальном случае, когда каждый из друзей Димы получит хотя бы одну наклейку, мы так же введем уравнение $x + y + z + w = n$, где $x, y, z, w \in \mathbb{N}$ — количества наклеек для каждого из друзей, $n \in \mathbb{N}$ — общее число наклеек. Так же установим ограничение $x, y, z, w \leqslant n - 3$ и уравнение $x + y + z = n - w$, эквивалентным данному, сведем к нахождению числа его решений к предыдущему случаю. Число решений этого уравнения при фиксированном $w$ равно $(n-w-2)(n-w-1)/2$. Суммируем по всем $w$ от 1 до $n - 3$
    \begin{equation*}
    \begin{aligned}
        \sum_{w=1}^{n-3}\frac{(n-w-2)(n-w-1)}{2} = \frac{1}{2}\sum_{w=1}^{n-3}(w^2 - (2n-3)w + n^2 - 3n + 2) = \\
        = \frac{1}{2}\left(\sum_{w=1}^{n-3}w^2 - (2n-3)\sum_{w=1}^{n-3}w + (n^2 - 3n + 2)\sum_{w=1}^{n-3}1\right) = \\
        = \frac{1}{2}\left(\frac{(n-3)(n-2)(2n-5)}{6} - \frac{(2n-3)(n-2)(n-3)}{2} + (n-1)(n-2)(n-3)\right) = \\
        = \frac{(n-3)(n-2)}{2}\left(\frac{2n-5}{6} - \frac{2n-3}{2} + n-1\right) = \\
        = \frac{(n - 3)(n - 2)(n - 1)}{6}
    \end{aligned}
    \end{equation*}
    В нашем случае $n = 9$ и число решений составляет 56. Поскольку из четырех друзей выбрать четверку человек можно только одним способом, то этот случай дает нам ровно 56 способов распределения наклеек.

    Таким образом, суммируя найденные количества способов, получаем, что у Димы есть ровно 220 способов раздать свои наклейки друзьям. Если же учитывать вариант, что ни один из друзей не получит наклейку, то способов будет на один больше, но эта ситуация не предусматривается.

    \underline{Ответ}: 220.
\end{document}